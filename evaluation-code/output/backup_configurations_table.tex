
\begin{figure}
    %  CNN
    \begin{subtable}{0.5\linewidth}
    \setlength{\tabcolsep}{2pt}
    \centering
    \fontsize{6}{6}\selectfont{
    \begin{tabular}{cc}
    {\resizebox{\columnwidth}{!}{
        \begin{tabular}{|p{\columnwidth}|}
          \hline
          {\textbf{CNN}} \\
          \hline
          \textit{Parameters}: $x$ kernels \\
          \hline
          1. Time-Distributed Reshape to shape (5, 3, 1) of each time frame.\\
          \hline
          2. Time-Distributed Conv2D-layer with $x$ kernels of shape (2, 2) followed by a Dropout layer with $p=0.25$\\
          \hline
          3. Another Time-Distributed Conv2D-layer with $x$ kernels of shape (2, 2) followed by a Dropout layer with $p=0.25$\\
          \hline
          4. A Dropout layer with $p=0.5$, a Flatten-Layer and a soft-max-activated Dense layer with 10 units, one for each output class.\\
          \hline
        \end{tabular} 
    }
    \\
    \\
    \resizebox{\columnwidth}{!}{
        \begin{tabular}{|p{\columnwidth}|}
          \hline
          {\textbf{CNN}} \\
          \hline
          \textit{Parameters}: $x$ kernels \\
          \hline
          1. Time-Distributed Reshape to shape (5, 3, 1) of each time frame.\\
          \hline
          2. Time-Distributed Conv2D-layer with $x$ kernels of shape (2, 2) followed by a Dropout layer with $p=0.25$\\
          \hline
          3. Another Time-Distributed Conv2D-layer with $x$ kernels of shape (2, 2) followed by a Dropout layer with $p=0.25$\\
          \hline
          4. A Dropout layer with $p=0.5$, a Flatten-Layer and a soft-max-activated Dense layer with 10 units, one for each output class.\\
          \hline
        \end{tabular} 
    }
    
    }
        &
        \resizebox{\columnwidth}{!}{
          \begin{tabular}{|p{\columnwidth}|}
          \hline
          {\textbf{RNN}} \\
          \hline
          \textit{Parameters}: $x$ units \\
          \hline
          1. A Time-Distributed Flatten layer to flatten each time frame.\\
          \hline
          2. A SimpleRNN-layer with $x$ units, followed by a Dropout layer with $p=0.25$.
          \hline
          3. a Flatten-Layer and a soft-max-activated Dense layer with 10 units, one for each output class.\\
          \hline
        \end{tabular}
    }
    \end{tabular}
    }
    \end{subtable}
    \caption{The configurations for each models that were tested}
    \label{fig:models}
\end{figure}

